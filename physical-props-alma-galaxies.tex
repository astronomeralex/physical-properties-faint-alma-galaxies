\documentclass[preprint,natbib209]{aastex}
\usepackage{url}
\usepackage{amsmath}
\usepackage{afterpage}
\slugcomment{\apj, submitted}
\shorttitle{Physically Properties of Serendipitous sub-mm Galaxies}
\shortauthors{Hagen et al.}
\newcommand{\eg}{e.g.}
\citestyle{apj}
\begin{document}

\title{Physically Properties of Faint Sub-millimeter Galaxies Serendipitously Discovered with ALMA}

\author{Alex Hagen\altaffilmark{1,2,3}}
\affil{Department of Astronomy \& Astrophysics, The Pennsylvania State 
University, 525 Davey Lab, University Park, PA 16802}
\email{hagen@psu.edu}

\author{Seiji Fujimoto, Masami Ouchi}
\affil{Institute for Cosmic Ray Research, The University of Tokyo, Kashiwa 277 8582, Japan}
\email{sfseiji@icrr.u-tokyo.ac.jp, ouchims@icrr.u-tokyo.ac.jp}

\altaffiltext{1}{Institute for Gravitation and the Cosmos, 
The Pennsylvania State University, University Park, PA 16802}
\altaffiltext{2}{NSF East Asia and Pacific Summer Institute Fellow 2015}
\altaffiltext{3}{Institute for Cosmic Ray Research, The University of Tokyo, Kashiwa 277 8582, Japan}

\begin{abstract} 

\end{abstract}
\keywords{galaxies: evolution -- galaxies: high-redshift -- cosmology: observations}

%\begin{figure}[htpb]
%	\centering
%	\includegraphics[scale=0.4]{fitting.pdf}
%	trim = left bottom right top
%	\caption{The given data and both fits. The Gaussian is a clearly a better fit than the Lorentzian}
%	\label{fit}
%\end{figure}
%\citep[e.g.,][]{verhamme12, yajima12, behrens14}

\section{Introduction}
\label{sec:intro}


We adopt the standard concordance cosmology of $h = 0.7$, $\Omega_m = 0.3$, 
$\Omega_\Lambda = 0.7$, and $\Omega_k = 0$ \citep{planck13}.

\section{Sample Selection} 
\label{sec:sample}



\section{Physical Properties}
\label{sec:analysis}



\section{Results}
\label{sec:discussion}



\section{Conclusion}
\label{sec:conclusion}



\acknowledgments
\section*{Acknowledgments}
The Institute for Gravitation and the Cosmos is 
supported by the Eberly College of Science and the Office of the Senior Vice
President for Research at the Pennsylvania 
State University. This research has made use of NASA's Astrophysics Data System 
and the python packages \texttt{IPython}, \texttt{AstroPy}, 
\texttt{NumPy}, \texttt{SciPy}, \texttt{scikit-learn}, and \texttt{matplotlib}
 \citep{ipython, astropy, numpy, scipy, scikit-learn, matplotlib}.


\bibliographystyle{apj}                       %% AASTeX
\bibliography{mybib}

\end{document}
