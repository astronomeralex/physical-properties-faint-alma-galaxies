\documentclass[preprint,natbib209]{aastex}
\usepackage{url}
\usepackage{amsmath}
\usepackage{afterpage}
\slugcomment{\apj, submitted}
\shorttitle{Physically Properties of Serendipitous sub-mm Galaxies}
\shortauthors{Hagen et al.}
\newcommand{\eg}{e.g.}
\citestyle{apj}
\begin{document}

\title{Heterogeneous Physical Properties of Serendipitously Discovered Galaxies with ALMA: 
Low-mass sub-millimeter galaxies and others at $0.6 < z < 3$}

\author{Alex Hagen\altaffilmark{1,2,3}}
\affil{Department of Astronomy \& Astrophysics, The Pennsylvania State 
University, 525 Davey Lab, University Park, PA 16802}
\email{hagen@psu.edu}

\author{Seiji Fujimoto, Masami Ouchi}
\affil{Institute for Cosmic Ray Research, The University of Tokyo, Kashiwa 277 8582, Japan}
\email{sfseiji@icrr.u-tokyo.ac.jp, ouchims@icrr.u-tokyo.ac.jp}

\altaffiltext{1}{Institute for Gravitation and the Cosmos, The Pennsylvania State University, University Park, PA 16802}
\altaffiltext{2}{NSF East Asia and Pacific Summer Institute Fellow 2015}
\altaffiltext{3}{Institute for Cosmic Ray Research, The University of Tokyo, Kashiwa 277 8582, Japan}

\begin{abstract} 

\end{abstract}
\keywords{galaxies: evolution -- galaxies: high-redshift -- cosmology: observations}

%\begin{figure}[htpb]
%	\centering
%	\includegraphics[scale=0.4]{fitting.pdf}
%	trim = left bottom right top
%	\caption{The given data and both fits. The Gaussian is a clearly a better fit than the Lorentzian}
%	\label{fit}
%\end{figure}
%\citep[e.g.,][]{verhamme12, yajima12, behrens14}

\section{Introduction}
\label{sec:intro}

This bias of these galaxies was significantly lower than that of traditional SMG and pBzK galaxies, 
and similar to those of sBzKs, LBGs, and LAEs. 

We adopt the standard concordance cosmology of $h = 0.7$, $\Omega_m = 0.3$, $\Omega_\Lambda = 0.7$, 
and $\Omega_k = 0$ \citep{planck13}.

\section{Sample Selection} 
\label{sec:sample}

Our sample comes from \cite{fujimoto15}, who studied serendipitously discovered sub-mm galaxies in 
120 deep ALMA pointings. This work achieved depths of 0.05 mJy at 1.2mm in the field and 0.02 mJy 
with the assistance of gravitational lensing in A1689; it discovered a total of 133 galaxies, of which 65 galaxies 
were in SXDS and A1689, which both have a rich set of multi-wavelength imaging. Of the 65 galaxies, 
17 are detected in the optical - NIR images and thus spectral energy distribution fitting can be applied to 
understand the physical properties of this sample. Two of these sources are in A1689 and the remaining
15 are in SXDS.

The photometry for our analysis comes from a variety of multi-wavelength surveys in these fields.
In SXDS we use the following bands \textit{u*} (Foucaud et al., in prep.), \textit{B, V, R, i', z'} \citep{furusawa08},
\textit{J, H, K} \citep[UKIDDS\footnotemark][]{lawrence07}


\footnotetext{The UKIDSS project is defined in \citet{lawrence07}. UKIDSS uses the UKIRT Wide Field Camera \citet[WFCAM][]{casali07}. The photometric system is described in \citet{hewett06}, and the calibration is described in \citet{hodgkin09}. The pipeline processing and science archive are described in \citet{hambly08}.}

%In the SXDS region, we use published photometry
%catalogs in 0.5-10 keV (Ueda et al. 2008), u
%?
%(Foucaud et al. in preparation), B, V , R, i
%?
%, z
%?
%(Furusawa et al. 2008), J, H, K (UKIDSS DR10; Almaini
%et al. in preparation), 3.6 ? 24 �m (Spitzer
%UKIDSS Ultra Deep Survey; SpUDS), 250 ? 500 �m
%(Herschel Multi-tiered Extragalactic Survey; HerMES,
%Oliver et al. 2012), and 1.4 GHz (Simpson et al. 2006)
%as well as the photometric redshift (zphot) catalog
%(Williams et al. 2009). In the A1689 region, we investigate
%published photometry catalogs of g475, r625,
%i775, z850, J110, H160 (Hubble Source Catalog; HSC),
%3.6 ? 24 �m (Spitzer Enhanced Imaging Products;
%SEIP), 250 ? 500 �m (HerMES), and the photometric
%redshift zphoto and spectroscopic redshift zspec catalogs
%(Limousin et al. 2007; Coe et al. 2010; Diego et al.
%2015).


\section{Physical Properties}
\label{sec:analysis}



\section{Results}
\label{sec:discussion}



\section{Conclusion}
\label{sec:conclusion}



\acknowledgments
\section*{Acknowledgments}
The Institute for Gravitation and the Cosmos is 
supported by the Eberly College of Science and the Office of the Senior Vice
President for Research at the Pennsylvania 
State University. This research has made use of NASA's Astrophysics Data System 
and the python packages \texttt{IPython}, \texttt{AstroPy}, 
\texttt{NumPy}, \texttt{SciPy}, \texttt{scikit-learn}, and \texttt{matplotlib}
 \citep{ipython, astropy, numpy, scipy, scikit-learn, matplotlib}.


\bibliographystyle{apj}                       %% AASTeX
\bibliography{mybib}

\end{document}
