\documentclass[preprint,natbib209]{aastex}
\usepackage{url}
\usepackage{amsmath}
\usepackage{afterpage}
\slugcomment{To be submitted to \apj\ or \mnras}
\shorttitle{Physically Properties of Serendipitous sub-mm Galaxies}
\shortauthors{Hagen et al.}
\newcommand{\eg}{e.g.}
\citestyle{apj}
\begin{document}

\title{Heterogeneous Physical Properties of Serendipitously Discovered Galaxies with ALMA: 
Low-mass sub-millimeter galaxies and others at $0.6 < z < 3$}

\author{Alex Hagen\altaffilmark{1,2,3}}
\affil{Department of Astronomy \& Astrophysics, The Pennsylvania State 
University, 525 Davey Lab, University Park, PA 16802}
\email{hagen@psu.edu}

\author{Seiji Fujimoto, Masami Ouchi}
\affil{Institute for Cosmic Ray Research, The University of Tokyo, Kashiwa 277 8582, Japan}
\email{sfseiji@icrr.u-tokyo.ac.jp, ouchims@icrr.u-tokyo.ac.jp}

\author{et al.}

\altaffiltext{1}{Institute for Gravitation and the Cosmos, The Pennsylvania State University, University Park, PA 16802}
\altaffiltext{2}{NSF East Asia and Pacific Summer Institute Fellow 2015}
\altaffiltext{3}{Institute for Cosmic Ray Research, The University of Tokyo, Kashiwa 277 8582, Japan}

\begin{abstract} 

\end{abstract}
\keywords{galaxies: evolution -- galaxies: high-redshift -- cosmology: observations}

%\begin{figure}[htpb]
%	\centering
%	\includegraphics[scale=0.4]{fitting.pdf}
%	trim = left bottom right top
%	\caption{The given data and both fits. The Gaussian is a clearly a better fit than the Lorentzian}
%	\label{fit}
%\end{figure}
%\citep[e.g.,][]{verhamme12, yajima12, behrens14}

\section{Introduction}
\label{sec:intro}

This bias of these galaxies was significantly lower than that of traditional SMG and pBzK galaxies, 
and similar to those of sBzKs, LBGs, and LAEs. 

We adopt the standard concordance cosmology of $h = 0.7$, $\Omega_m = 0.3$, $\Omega_\Lambda = 0.7$, 
and $\Omega_k = 0$ \citep{planck13}.

\section{Sample Selection} 
\label{sec:sample}

Our sample comes from \citet{fujimoto15}, hereafter F15, who studied serendipitously discovered sub-mm galaxies in 
120 deep ALMA pointings pulled from the ALMA archive. This work achieved depths of 0.05 mJy at 1.2mm in the field and 0.02 mJy 
with the assistance of gravitational lensing in A1689; it discovered a total of 133 galaxies. Many of the ALMA observations in F15 were not
in deep fields with ancillary photometry sufficient to identify counterparts. Thus, F15 focused on identifying counterparts in Subaru XMM-Newton Deep 
Survey \citep[SXDS;][]{furusawa08} and Abel 1689 (A1689), both of which are covered by multi-wavelength surveys. 
Out of 65 serendipitous ALMA sources, F15 identified 17 counterparts -- 2 in A1689 and 15 in SXDS. 

% add in cosmos counterparts
To ensure were were not missing any counterparts, we searched for counterparts for each galaxy in The NASA Extragalactic Data (NED), and
the \textit{Hubble} Source Catalog (HSC; Whitmore et al., in prep). This led us to discover YY new optical counterparts, 
mostly in the COMOS field \citep{COSMOS}. %have a table of new counterparts? and maybe a table of photometry?
We identified an optical counterpart for F15 source ID 7 originally found in the HSC and the Sloan Digital Sky Survey \citep[SDSS;][]{york2000} 
DR12 \citep{alam15} imaging data \citep{fukugita96, gunn98, hogg01, smith02, pier03, ivezic04, gunn06, tucker06, padmanabhan08, doi10}.

The photometry for the counterparts identified in F15 comes from a variety of multi-wavelength surveys in SXDS and A1689.
In SXDS we use the following bands \textit{u*} (Foucaud et al., in prep.), \textit{B, V, R, i', z'} \citep{furusawa08},
\textit{J, H, K} \citep[UKIDSS;\footnotemark][]{lawrence07} 3.6 - 24 $\mu$m \citep[SWIRE, SpUDS, SEDS;][]{lonsdale03, ashby13}
250 - 500 $\mu$m \citep[HerMES;][]{oliver12, smith12, wang14} and 1.4 GHz \citep{simpson06}.
In A1689 we use \textit{g$_{475}$, r$_{625}$, i$_{775}$, z$_{850}$, J$_{110}$} \& \textit{H}$_{160}$ from the HSC,
 3.6 - 24 $\mu$m from \textit{Spitzer} Enhanced Imaging Products (SEIP), and 250 - 500 $\mu$m (HerMES).

\footnotetext{The UKIDSS project is defined in \citet{lawrence07}. UKIDSS uses the UKIRT 
Wide Field Camera \citet[WFCAM][]{casali07}. The photometric system is described in 
\citet{hewett06}, and the calibration is described in \citet{hodgkin09}. 
The pipeline processing and science archive are described in \citet{hambly08}.}

\section{Physical Properties}
\label{sec:analysis}

%talk about how important it is to have consistency in whole fit -- energy extinguished from dust heats dust, so there's energy balance

We used two different software packages to model the SEDs of our galaxies, \texttt{Multi-wavelength Analysis of Galaxy Physical Properties} (MAGPHYS) \citep{dacunha08, dacunha15} and \texttt{Flexible Stellar Population Synthesis} (FSPS) \citep{conroy09, conroy10}. Both of these packages generate SEDs with inputs of physical parameters and stellar isochrones; the best-fit physical parameters and their uncertainties are found by maximizing the likelihood, using a Bayesian approach by taking priors into account when an informed prior can be found. MAGPHYS builds up the marginalized likelihood for each parameter by comparing the data to all combinations of models in a grid search. For FSPS we used a the \texttt{emcee} Markov Chain Monte Carlo package \citep{emcee} coupled with the \texttt{python-fsps} interface to \texttt{FSPS} \citep{python-fsps}. For more detail on the software setup for FSPS, see Brooks et al. (2016). 

In our use of \texttt{FSPS}, we found that the outputs could not be used for further analysis, as we hit an edge in the parameter distribution for the strength of the radiation field. The dust emission models used in \texttt{FSPS}, from \cite{draine07}, allow the strength of the radiation field go to 25 times the Milky Way level. The output distributions for all galaxies consistently hit the edge of this prior, meaning that our results our biased as there are well-fitting models that lie outside the range of the \texttt{FSPS} software. While we do not use the physical parameter outputs, there is still a lesson to be learned here -- the radiation field intensities in these galaxies are very high compared to the Milky Way. This is consistent with their compact nature and star formation rates. We also hope that this finding provides motivation to increase the parameter range in the dust emission models in \texttt{FSPS}.

\subsection{MAGPHYS Analysis}



\section{Results}
\label{sec:discussion}



\section{Conclusion}
\label{sec:conclusion}



\acknowledgments
\section*{Acknowledgments}
Based in part on observations made with the NASA/ESA Hubble Space Telescope, 
and obtained from the Hubble Legacy Archive, which is a collaboration between
the Space Telescope Science Institute (STScI/NASA), the Space Telescope 
European Coordinating Facility (ST-ECF/ESAC/ESA) and the 
Canadian Astronomy Data Centre (CADC/NRC/CSA). 
This work is based in part on observations made with the Spitzer Space Telescope, 
which is operated by the Jet Propulsion Laboratory, California Institute of Technology under a contract with NASA.
This research has made use of data from HerMES project (\url{http://hermes.sussex.ac.uk/}). 
HerMES is a Herschel Key Program utilizing Guaranteed Time from the SPIRE instrument team,
ESAC scientists and a mission scientist.
The HerMES data was accessed through the Herschel Database in 
Marseille (HeDaM - \url{http://hedam.lam.fr}) operated by CeSAM and
hosted by the Laboratoire d'Astrophysique de Marseille.

Funding for the SDSS and SDSS-II has been provided by the Alfred P. Sloan Foundation, 
the Participating Institutions, the National Science Foundation, the U.S. Department of Energy, 
the National Aeronautics and Space Administration, the Japanese Monbukagakusho, 
the Max Planck Society, and the Higher Education Funding Council for England. 
The SDSS Web Site is \url{http://www.sdss.org/.}
Funding for SDSS-III has been provided by the Alfred P. Sloan Foundation, the Participating Institutions, 
the National Science Foundation, and the U.S. Department of Energy Office of Science. 
The SDSS-III web site is \url{http://www.sdss3.org/}.

This research has made use of NASA's Astrophysics Data System.
This research has made use of the NASA/ IPAC Infrared Science Archive, 
which is operated by the Jet Propulsion Laboratory, California Institute of Technology, 
under contract with the National Aeronautics and Space Administration.
This research has made use of the NASA/IPAC Extragalactic Database (NED), 
which is operated by the Jet Propulsion Laboratory, California Institute of Technology, 
under contract with the National Aeronautics and Space Administration.
This research has made use of the VizieR catalogue access tool, CDS,
Strasbourg, France. The original description of the VizieR service was
published in \cite{vizier}.

The Institute for Gravitation and the Cosmos is 
supported by the Eberly College of Science and the Office of the Senior Vice
President for Research at the Pennsylvania 
State University. This research has made use of NASA's Astrophysics Data System 
and the python packages \texttt{IPython}, \texttt{AstroPy}, 
\texttt{NumPy}, \texttt{SciPy}, \texttt{scikit-learn}, and \texttt{matplotlib}
 \citep{ipython, astropy, numpy, scipy, scikit-learn, matplotlib}.


\bibliographystyle{apj}                       %% AASTeX
\bibliography{mybib}

\end{document}
